% Author: Dominik Harmim <xharmi00@stud.fit.vutbr.cz>
% Author: Vojtěch Hertl <xhertl04@stud.fit.vutbr.cz>


\documentclass[a4paper, 11pt]{article}


\usepackage[czech]{babel}
\usepackage[utf8]{inputenc}
\usepackage[left=2cm, top=3cm, text={17cm, 24cm}]{geometry}
\usepackage{times}
\usepackage{graphicx}
\usepackage[unicode]{hyperref}
\hypersetup{
	colorlinks = true,
	hypertexnames = false,
	citecolor = red
}


\begin{document}
	%%%%%%%%%%%%%%%%%%%%%%%%%%%%%%%% Titulní stránka %%%%%%%%%%%%%%%%%%%%%%%%%%%
	\begin{titlepage}
		\begin{center}
			\includegraphics[width=0.77\linewidth]{inc/FIT_logo.pdf} \\

			\vspace{\stretch{0.382}}

			\Huge{Simulační studie} \\
			\LARGE{\textbf{Rozvoz jídla firmou Freshbox}} \\
			\Large{Tým: ModelX} \\
			\Large{Varianta 2: Doprava zboží nebo osob}

			\vspace{\stretch{0.618}}
		\end{center}

		\begin{minipage}{0.5 \textwidth}
			{\Large \today}
		\end{minipage}
		\hfill
		\begin{minipage}[r]{0.5 \textwidth}
			\Large
			\begin{tabular}{l l}
				\textbf{Dominik Harmim} & \textbf{(xharmi00)} \\
				Vojtěch Hertl & (xhertl04) \\
			\end{tabular}
		\end{minipage}
	\end{titlepage}



	%%%%%%%%%%%%%%%%%%%%%%%%%%%%%%%% Obsah %%%%%%%%%%%%%%%%%%%%%%%%%%%%%%%%%%%%%
	\pagenumbering{roman}
	\setcounter{page}{1}
	\tableofcontents
	\clearpage



	%%%%%%%%%%%%%%%%%%%%%%%%%%%%%%%% Úvod %%%%%%%%%%%%%%%%%%%%%%%%%%%%%%%%%%%%%%
	\pagenumbering{arabic}
	\setcounter{page}{1}

	\section{Úvod}

	V~této práci je řešen proces sestavování modelu \cite[snímek 7]{IMS_slides}
	pro rozvoz jídla po Brně firmou Freshbox \cite{Freshbox} a~jeho následná
	simulace \cite[snímek 33]{IMS_slides}. Díky tomuto modelu a~simulačním
	experimentům \cite[snímek 9]{IMS_slides} nad ním je možno pozorovat
	efektivitu a~přínos v~různých podmínkách. Smyslem experimentů je zjistit,
	jak kvalitně navržený je systém \cite[snímek 18]{IMS_slides} a~zda by se
	změnou některého z~ovlivňujících faktorů mohl zdokonalit. V~reálném
	systému může být obtížné a~finančně nákladné tyto ovlivňující faktory
	měnit a~zjišťovat, jak se bude systém chovat, proto je vhodné získat
	nové znalosti o~reálném systému použitím principů modelování
	a~simulace \cite[snímek 9]{IMS_slides}.


	\subsection{Autoři, zdroje}

	Projekt vypracovali studenti Dominik Harmim a~Vojtěch Hertl z~FIT VUT
	v~Brně.

	K~technické části této práce bylo využito zdrojů z~kursu Modelování
	a~simulace na FIT VUT v~Brně. Jako zdroj k~faktům sloužily webové stránky
	firmy Freshbox a~také vedoucí této firmy, Mgr. Silvie Obadalová\\
	(\texttt{obadalova@freshbox.cz}).


	\subsection{Ověření validity}

	Ověřování validity \cite[snímek 37]{Freshbox} probíhalo telefonicky
	a~elektronicky s~vedoucí firmy Freshox, magistrou Obadalovou. Na základě
	této komunikace byla získána všechna data potřebná k~experimentálnímu
	ověřování validity modelu. Statistická data byla extrahována z~naměřených
	statistik firmy Freshbox. Validita byla také ověřena pomocí experimentů
	a~srovnáním s~realitou.



	%%%%%%%%%%%%%%%% Rozbor tématu a použitých metod/technologií %%%%%%%%%%%%%%%
	\section{Rozbor tématu a~použitých metod/technologií}

	Všechna použitá fakta jsou zprůměrována ze všech získaných informací. \\

	Zákazníci mají předem objednaná jidla od firmy Freshbox, která tato jídla
	každý den od 6:30 hodin do 12:30 hodin (tj. 6 hodin) rozváží zákazníkům po
	Brně a~okolí. Firma Freshbox rozváží jidlo 16 auty, přičemž jedno auto
	je schopné naložit maximálně 500 jídel. Každý den se rozváží
	průměrně 21\,200 jídel $\pm$~1\,000 jídel. Firma má na začátku rozvozu
	již všechna jídla připravena a~v~6:30 hodin se připraví všechna auta,
	do kterých se naloží maximální počet jídel, který je dán maximální
	kapacitou auta. Naložení jednoho auta průměrně trvá 11 minut
	$\pm$~3 minuty. Rozvoz všech jídel jednoho auta trvá průměrně 97 minut
	$\pm$~12 minut. Při tomto rozvozu každé auto urazí průměrně 43 km
	$\pm$ 8 km. Pro rozvoz se požívají auta Volkswagen Caddy 1.9 TDI.
	Tato auta mají spotřebu
	7,7\,l/100\,km\footnote{\url{https://www.auto-data.net/en/volkswagen-caddy-maxi-life-typ-2k-1.9-tdi-105hp-8855}}
	nafty při městském provozu. Když auto rozveze všechna
	naložená jídla, vrátí se na pobočku Freshbox, aby se mohla naložit další
	jídla. Samotné nakládání jídel provádí sám řadič daného auta. Tento proces
	se opakuje tak dlouho, dokud nejsou rozvezena všechna jídla. Jeden zákazník
	(právnická nebo fyzická osoba) si samozřejmě může objednat více jídel na
	jedno místo doručení, což se typicky děje, protože pravidelnými zákazníky
	jsou firmy, které si objednávají denně řádově desítky jídel.


	\subsection{Použité postupy}

	Pro vytvoření modelu byl použit programovací jazyk C++ za podpory
	simulační knihovny SIMLIB \cite{SIMLIB}. Tyto technologie jsou ideální pro
	řešení zadaného problému, jelikož poskytují všechna potřebná rozhraní
	k~implementaci modelu. Další výhodou je, že se jedná o~otevřený software,
	jsou to multi-platformní technologie a~poměrně jednuduše se používají,
	nejedná se o~nic zbytečně těžkopádného. Dále byly použity postupy popsané
	v~textech ke kursu Modelování a simulace na FIT VUT v~Brně \cite{IMS_slides}
	k~vytváření Petriho sítě \cite[snímek 123]{IMS_slides} a~samotnému
	programování za použití knihovny SIMLIB.


	\subsection{Popis původu použitých metod/technologií}

	Použili jsme standardní třídy a~funkce jazyka
	C++\footnote{\url{https://cppreference.com/w/cpp}}.
	Drželi jsme se standardu C++14. Využívali jsme monžosti oběktově
	orientovaného návrhu.

	Pro překlad zdrojových souborů byly použity nástroje
	CMake\footnote{\url{https://cmake.org}}
	a~GNU Make\footnote{\url{https://www.gnu.org/software/make}}.

	Knihovna SIMLIB byla získána z~oficálních stránek tohoto
	nástroje\footnote{\url{http://www.fit.vutbr.cz/~peringer/SIMLIB}}.
	Použili jsme nejnovější dostupnou verzi (ke dni \today), tj 3.07.
	Autory tohoto nástroje jsou Petr Peringer, David Leska a David Martinek,
	viz \cite{SIMLIB}. Pro účely vytvoření našeho simulačního modelu jsme
	používali standardní nástroje a~rohraní této knihovny.



	%%%%%%%%%%%%%%%%%%%%%%%%%%%%%%%% Koncepce modelu %%%%%%%%%%%%%%%%%%%%%%%%%%%
	\section{Koncepce modelu}

	V~této sekci se zpracovává návrh konceptuálního modelu[snímek 48]\cite{IMS_slides} nad
	systémem, který je brán ze své podstaty jako systém hromadné obsluhy. Při vytváření
	je potřeba vybrat ze všech údajů ty podstatné informace pro model. Z~rozboru tématu 
	plyne, že je důležité namodelovat všecho, co souvisí se samotným rozvozem. Díky 
	skutečnosti, že všechny jednotlivé údaje jsou zprůměrovány, je zapotřebí simulovat průběh
	jednoho dne, přičemž se samozřejmě může den ode dne nepatrně lišit.	Na oba proměnné 
	časové údaje při modelování se tedy použije rovnoměrné rozdělení[snímek 89]\cite{IMS_slides}
	s~vyhovující odchylkou. Aby se model zjednodušil, průměrný počet jídel, který se 
	každý den rozváží, se zaokrouhlí, aby byl dělitelný maximální kapacitou aut. Na 
	validitu to má nepatrný	vliv až zanedbatelný jelikož jsou údaje zprůměrovány. Dále 
	značka auta, spotřeba paliva a~cena nejsou pro model důležitá, tyto informace budou 
	použity při	zefektivňování systému. Přesto, že tato situace reálně často, díky již 
	nabraným zkušenostem z reality, nenastává, pro lepší experimentování je přidán druhý 
	koncový stav, který znamená, že směna skončí dříve než jsou rozvezena všechna jídla, 
	tedy nějaká	jídla zbydou na skladě. Tento stav značí úspěšné dokončení směny, protože 
	se stihlo rozvést všechno jídlo v časovém limitu. Předpokládá se také, že auto, které 
	započne ještě během pracovní směny svůj cyklus, ho celý dokončí. Ovšem pokud mezitím 
	skončila směna, znamená to, že byl rozvoz neúspěšný z časového hlediska.


	\subsection{Popis konceptuálního modelu}

	Model se skládá ze~dvou hlavních větví. První značí samotný průběh
	rozvozu jídel a~druhá časovač. Druhá z~větví jen určuje, jak dlouho
	probíhá pracovní směna, to je 6 hodin, a~jakmile směna skončí, skončí
	rozvoz jídel. První větev má dvě vstupní proměnné - počet aut a~počet
	jídel. Auto zde slouží jako obslužná linka, kde pokud je některé volné
	a~připravené na rozvoz a~zároveň jsou ještě nějaká jídla na skladě,
	začnou se nakládat. Pokud je však některé z~aut připraveno na rozvoz,
	ale již byla všechna jídla rozvezena, skončí pracovní směna. Po
	naložení jídel se auto vydá na cestu a všechna jídla rozveze. Po ukončení
	této činnosti je auto zase volné a~připraveno k dalšímu použití.


	\subsection{Forma konceptuálního modelu}

	Model je vizualizován pomocí Petriho sítě v~příloze
	\ref{appendix:petri_net} a doplněn informacemi a legendou.



	%%%%%%%%%%%%%%%%%%% Architektura simulačního modelu %%%%%%%%%%%%%%%%%%%%%%%%
	\section{Architektura simulačního modelu}


	\subsection{Mapování konceptuálního modulu do simulačního modelu}



	%%%%%%%%%%%%%%% Podstata simulačních experimentů a jejich průběh %%%%%%%%%%%
	\section{Podstata simulačních experimentů a~jejich průběh}


	\subsection{Postup experimentování}


	\subsection{Experimenty}


	\subsection{Závěry experimentů}



	%%%%%%%%%%%%%%%%% Shrnutí simulačních experimentů a závěr %%%%%%%%%%%%%%%%%%
	\section{Shrnutí simulačních experimentů a~závěr}



	%%%%%%%%%%%%%%%%%%%%%%%%%%%%%%%% Citace %%%%%%%%%%%%%%%%%%%%%%%%%%%%%%%%%%%%
	\clearpage
	\bibliographystyle{czechiso}
	\renewcommand{\refname}{Literatura}
	\bibliography{documentation}



	%%%%%%%%%%%%%%%%%%%%%%%%%%%%%%%% Přílohy %%%%%%%%%%%%%%%%%%%%%%%%%%%%%%%%%%%
	\clearpage
	\appendix

	\section{Petriho síť}
	\label{appendix:petri_net}
	\begin{figure}[!ht]
		\centering
		\vspace{-1.2cm}
		\includegraphics[width=0.95\linewidth]{inc/petri_net.pdf}
		\caption{Petriho síť}
		\label{figure:petri_net}
	\end{figure}
\end{document}
