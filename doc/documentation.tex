% Author: Dominik Harmim <xharmi00@stud.fit.vutbr.cz>
% Author: Vojtěch Hertl <xhertl04@stud.fit.vutbr.cz>


\documentclass[a4paper, 11pt]{article}


\usepackage[czech]{babel}
\usepackage[utf8]{inputenc}
\usepackage[left=2cm, top=3cm, text={17cm, 24cm}]{geometry}
\usepackage{times}
\usepackage{graphicx}
\usepackage[hyphens]{url}
\usepackage[unicode, colorlinks, hypertexnames=false, citecolor=red]{hyperref}
\usepackage[czech, boxed]{algorithm2e}
\usepackage[perpage]{footmisc}


\begin{document}
	%%%%%%%%%%%%%%%%%%%%%%%%%%%%%%%% Titulní stránka %%%%%%%%%%%%%%%%%%%%%%%%%%%
	\begin{titlepage}
		\begin{center}
			\includegraphics[width=0.77 \linewidth]{inc/FIT_logo.pdf}

			\vspace{\stretch{0.382}}

			\Huge{Simulační studie} \\
			\LARGE{\textbf{Rozvoz jídla firmou Freshbox}} \\
			\Large{Tým: ModelX} \\
			\Large{Varianta 2: Doprava zboží nebo osob}

			\vspace{\stretch{0.618}}
		\end{center}

		\begin{minipage}{0.5 \textwidth}
			\Large
			\today
		\end{minipage}
		\hfill
		\begin{minipage}[r]{0.5 \textwidth}
			\Large
			\begin{tabular}{ll}
				\textbf{Dominik Harmim} & \textbf{(xharmi00)} \\
				Vojtěch Hertl & (xhertl04)
			\end{tabular}
		\end{minipage}
	\end{titlepage}



	%%%%%%%%%%%%%%%%%%%%%%%%%%%%%%%% Obsah %%%%%%%%%%%%%%%%%%%%%%%%%%%%%%%%%%%%%
	\clearpage
	\pagenumbering{roman}
	\setcounter{page}{1}
	\tableofcontents



	%%%%%%%%%%%%%%%%%%%%%%%%%%%%%%%% Úvod %%%%%%%%%%%%%%%%%%%%%%%%%%%%%%%%%%%%%%
	\clearpage
	\pagenumbering{arabic}
	\setcounter{page}{1}

	\section{Úvod}

	V~této práci je řešen proces sestavování modelu \cite[snímek 7]{IMS_slides}
	pro rozvoz jídla po Brně firmou Freshbox \cite{Freshbox} a~jeho následná
	simulace \cite[snímek 33]{IMS_slides}. Díky tomuto modelu a~simulačním
	experimentům \cite[snímek 9]{IMS_slides} nad ním je možno pozorovat
	efektivitu a~přínos v~různých podmínkách. Smyslem experimentů je zjistit,
	jak kvalitně navržený je systém \cite[snímek 18]{IMS_slides} a~zda by se
	změnou některého z~ovlivňujících faktorů mohl zdokonalit. V~reálném
	systému může být obtížné a~finančně nákladné tyto ovlivňující faktory
	měnit a~zjišťovat, jak se bude systém chovat, proto je vhodné získat
	nové znalosti o~reálném systému použitím principů modelování
	a~simulace \cite[snímek 9]{IMS_slides}.


	\subsection{Autoři, zdroje}

	Projekt vypracovali studenti Dominik Harmim a~Vojtěch Hertl z~FIT VUT
	v~Brně.

	K~technické části této práce bylo využito zdrojů z~kursu Modelování
	a~simulace na FIT VUT v~Brně. Jako zdroj k~faktům sloužily webové stránky
	firmy Freshbox a~také vedoucí této firmy, Mgr. Silvie Obadalová \\
	(\texttt{obadalova@freshbox.cz}).


	\subsection{Ověření validity}

	Ověřování validity \cite[snímek 37]{IMS_slides} probíhalo telefonicky
	a~elektronicky s~vedoucí firmy Freshox, magistrou Obadalovou. Na základě
	této komunikace byla získána všechna data potřebná k~experimentálnímu
	ověřování validity modelu. Statistická data byla extrahována z~naměřených
	statistik firmy Freshbox. Validita byla také ověřena pomocí experimentů
	a~srovnáním s~realitou.



	%%%%%%%%%%%%%%%% Rozbor tématu a~použitých metod/technologií %%%%%%%%%%%%%%%
	\section{Rozbor tématu a~použitých metod/technologií}

	Všechna použitá fakta jsou zprůměrována ze všech získaných informací. \\

	Zákazníci mají předem objednaná jidla od firmy Freshbox, která tato jídla
	každý den od 6:30 hodin do 12:30 hodin (tj. 6 hodin) rozváží zákazníkům po
	Brně a~okolí. Firma Freshbox rozváží jidlo 16 auty, přičemž jedno auto
	je schopno pojmout maximálně 500 jídel. Každý den se rozváží
	průměrně 21\,200 jídel $\pm$\,1\,000 jídel včetně polévek, salátů
	a~dezertů. Firma má na začátku rozvozu již všechna jídla připravena
	a~v~6:30 hodin se připraví všechna auta, do kterých se naloží maximální
	počet jídel, který je dán maximální kapacitou auta. Naložení jednoho auta
	průměrně trvá 11 minut $\pm$\,3 minuty. Rozvoz všech jídel jednoho auta
	trvá průměrně 97 minut $\pm$\,12 minut. Při tomto rozvozu každé auto urazí
	průměrně 43\,km $\pm$\,8\,km. Pro rozvoz se požívají auta Volkswagen Caddy
	1.9 TDI. Tato auta mají spotřebu 7,7 l/100\,km\footnote{
		\url{https://www.auto-data.net/en/volkswagen-caddy-maxi-life-typ-2k-1.9-tdi-105hp-8855}
	} nafty při městském provozu. Když auto rozveze všechna naložená jídla,
	vrátí se na pobočku Freshbox, aby se mohla naložit další jídla. Samotné
	nakládání jídel provádí sám řidič daného auta. Tento proces se opakuje
	tak dlouho, dokud nejsou rozvezena všechna jídla. Jeden zákazník
	(právnická nebo fyzická osoba) si samozřejmě může objednat více jídel na
	jedno místo doručení, což se typicky děje, protože pravidelnými zákazníky
	jsou firmy, které si objednávají denně řádově desítky jídel.


	\subsection{Použité postupy}

	Pro vytvoření modelu byl použit programovací jazyk C++ za podpory
	simulační knihovny SIMLIB \cite{SIMLIB}. Tyto technologie jsou ideální pro
	řešení zadaného problému, jelikož poskytují všechna potřebná rozhraní
	k~implementaci modelu. Další výhodou je, že se jedná o~otevřený software,
	jsou to multi-platformní technologie a~poměrně jednuduše se používají,
	nejedná se o~nic zbytečně těžkopádného. Dále byly použity postupy popsané
	v~textech ke kursu Modelování a~simulace na FIT VUT v~Brně \cite{IMS_slides}
	k~vytváření Petriho sítě \cite[snímek 123]{IMS_slides} a~samotnému
	programování za použití knihovny SIMLIB.


	\subsection{Popis původu použitých metod/technologií}

	Byly použity standardní třídy a~funkce jazyka
	C++\footnote{\url{https://cppreference.com/w/cpp}}, přičemž
	byl dodržen standard C++14. Využívá se také monžosti oběktově
	orientovaného návrhu. Pro překlad zdrojových souborů byly použity nástroje
	CMake\footnote{\url{https://cmake.org}}
	a~GNU Make\footnote{\url{https://www.gnu.org/software/make}}.

	Knihovna SIMLIB byla získána z~oficálních stránek tohoto
	nástroje\footnote{\url{http://www.fit.vutbr.cz/~peringer/SIMLIB}}.
	Použita byla nejnovější dostupná verze (ke dni \today), tj 3.07.
	Autory tohoto nástroje jsou Petr Peringer, David Leska a~David Martinek,
	viz \cite{SIMLIB}. Pro účely vytvoření simulačního modelu
	\cite[snímek 44]{IMS_slides} se využilo standardních nástrojů
	a~rohraní této knihovny.



	%%%%%%%%%%%%%%%%%%%%%%%%%%%%%%%% Koncepce modelu %%%%%%%%%%%%%%%%%%%%%%%%%%%
	\section{Koncepce modelu}

	V~této sekci se zpracovává návrh konceptuálního (abstraktního) modelu
	\cite[snímek 48]{IMS_slides} nad systémem, který je brán ze své podstaty
	jako systém hromadné obsluhy \cite[snímek 136]{IMS_slides}. Při vytváření
	je potřeba vybrat ze všech údajů ty podstatné informace pro model.
	Z~rozboru tématu plyne, že je důležité namodelovat všechno, co souvisí se
	samotným rozvozem. Díky skutečnosti, že všechny jednotlivé údaje jsou
	zprůměrovány, je zapotřebí simulovat průběh jen jednoho dne, přičemž se
	samozřejmě může den ode dne nepatrně lišit.	Na oba proměnné časové údaje
	při modelování se tedy použije rovnoměrné rozdělení
	\cite[snímek 89]{IMS_slides} s~vyhovující odchylkou. Aby se model
	zjednodušil, průměrný počet jídel, který se každý den rozváží, se
	zaokrouhlí, aby byl dělitelný maximální kapacitou aut. Na validitu to má
	nepatrný vliv, až zanedbatelný, jelikož jsou údaje zprůměrovány. Dále
	značka auta, spotřeba paliva a~cena nejsou pro model důležitá, tyto
	informace budou použity při experimentech se snahou o~zefektivnění systému.
	Přesto, že tato situace reálně často, díky již nabraným zkušenostem
	z~reality, nenastává, pro lepší experimentování je přidán druhý koncový
	stav, který znamená, že směna skončí dříve, než jsou rozvezena všechna
	jídla, tedy nějaká jídla zbydou na skladě. Tento stav značí neúspěšné
	dokončení směny, protože se nestihlo rozvézt všechno jídlo v~časovém
	limitu. Předpokládá se také, že auto, které započne ještě během pracovní
	směny svůj cyklus, ho celý dokončí. Ovšem pokud mezitím skončila směna,
	znamená to, že byl rozvoz neúspěšný z~časového hlediska.


	\subsection{Popis konceptuálního modelu}
	\label{section:conceptual_model_desc}

	Model, viz příloha \ref{appendix:petri_net}, se skládá ze~dvou hlavních
	větví. První značí samotný průběh rozvozu jídel a~druhá časovač. Druhá
	z~větví jen určuje, jak dlouho probíhá pracovní směna, to je 6 hodin,
	a~jakmile směna skončí, skončí rozvoz jídel. První větev má dvě vstupní
	proměnné -- počet aut a~počet jídel. Auto zde slouží jako obslužná linka,
	kde pokud je některé volné a~připravené na rozvoz a~zároveň jsou ještě
	nějaká jídla na skladě, začnou se nakládat. Pokud ale již byla všechna
	jídla rozvezena a~všechna auta jsou připravena na rozvoz, skončí pracovní
	směna. Po naložení jídel se auto vydá na cestu a~všechna jídla rozveze.
	Po ukončení této činnosti je auto zase volné a~připraveno k~dalšímu použití.


	\subsection{Forma konceptuálního modelu}

	Model je vizualizován pomocí Petriho sítě v~příloze
	\ref{appendix:petri_net} a~doplněn informacemi a~legendou.



	%%%%%%%%%%%%%%%%%%% Architektura simulačního modelu %%%%%%%%%%%%%%%%%%%%%%%%
	\section{Architektura simulačního modelu}
	\label{section:simulation_model_architecture}

	Při spuštění simulačního modelu \cite[snímek 44]{IMS_slides} se 5krát
	po sobě spustí simulační experiment se zadanými parametry,
	viz kapitola \ref{section:run_simulation_model}. Po dokončení
	běhu simulace se na standardní výstup vypíší informace o~provedeném
	experimentu. Před začátkem každého z~experimentů se vypíše modelový
	čas \cite[snímek 21]{IMS_slides} začátku experimentu, počet dostupných
	aut pro rozvážení jídla a~počet jídla, které je potřeba rozvézst. Na
	konci každého z~experimentů se vypíše modelový čas konce experimentu,
	počet jídel, které se nestihly rozvézt, celková auty ujetá vzdálenost,
	celková spotřeba pohonných hmot všech aut, informace o~skladu
	\cite[snímek 184]{IMS_slides} aut a~statistiky ohledně času nakládání aut,
	době trvání rozvozu jídel, ujeté vzdálenosti a~spotřebě aut.

	Před spuštěním každého z~experimentů se inicializuje modelový čas tak, aby
	doba trvání experimentu odpovídala jednomu dni v~reálném čase
	\cite[snímek 21]{IMS_slides}, protože modelujeme jeden den rozvozu jídla.
	Jedna časová jednotka modelového času odpovídá jedné minutě v~reálném
	čase.

	Spuštění experimentu v~simulačním modelu znamená vytvořit a~aktivovat
	proces \cite[snímek 121]{IMS_slides}, který reprezentuje pracovní směnu
	a~následně spustit samotnou simulaci. Tento proces reprezentující
	pracovní směnu lze vytvořit s~parametry, které mění chování simulace.
	Tyto parametry jsou měněny při spouštění jednotlivých experimentů. Jedná
	se o~parametr značící počet aut pro rozvoz jídel a~průměrný počet jídel
	pro rozvoz.

	Na algoritmu \ref{algorithm:workshift} je znázorněno chování procesu
	pracovní směny a~na algoritmu \ref{algorithm:car} je znázorněno chování
	procesu auta. \\

	\begin{algorithm}[ht]
		\SetKw{Break}{break}

		vytvoření události reprezentující časovač, který po určité době
		ukoční směnu\;
		\While{jsou na skladě jídla připravena k~rozvozu}
		{
			čekání na volné auto a~jeho zabrání\;
			\eIf{na skladě už nejsou žádná jídla}
			{
				vrácení auta do skladu aut\;
				\Break\;
			}
			{
				odebrání jídel ze skladu, které budou naloženy do auta\;
				vytvoření a~aktivavce procesu auta\;
			}
		}
		čekání, až budou všechna auta ve skladu\;
		ukončení směny\;

		\caption{Chování procesu pracovní směny}
		\label{algorithm:workshift}
	\end{algorithm}

	\begin{algorithm}[ht]
		čekání na naložení auta\;
		čekání na rozvezení všech jídel\;
		vrácení auta do skladu aut\;

		\caption{Chování procesu auta}
		\label{algorithm:car}
	\end{algorithm}


	\subsection{Mapování konceptuálního modulu do simulačního modelu}

	Jak je popsáno v~kapitole \ref{section:conceptual_model_desc}
	a~znázorněno v~Petriho síti v~příloze \ref{appendix:petri_net}, časovač,
	který ukončí směnu po uplynutí konce směny, je v~simulačním modelu
	implementován jako událost \cite[snímek 169]{IMS_slides}, která
	při jejím vytvoření naplánuje sama sebe na čas, kdy má pracovní směna
	skončit. V~popisu chování této události je potom implementováno
	ukončení procesu pracovní směny. Této události v~simulačním modelu
	odpovídá třída \texttt{WorkShiftTimer}.

	Pracovní směna je v~simulačním modelu implementována jako proces, který
	při svém vytvoření vytvoří a~inicializuje sklad
	\cite[snímek 184]{IMS_slides} aut a~sdílenou proměnnou, která reprezentuje
	momentální počet jídel pro rozvoz. Chování tohoto procesu
	je znázorněno algoritmem \ref{algorithm:workshift}. Tomuto procesu
	v~simulačním modelu odpovídá třída \texttt{WorkShift}.

	Rozvoz jídla auty je v~simulačním modelu implementován procesem, jehož
	chování je znázorněno na algoritmu \ref{algorithm:car}. Tomuto procesu
	v~simulačním moodelu odpovídá třída \texttt{Car}.


	\subsection{Spouštění simulačního modelu}
	\label{section:run_simulation_model}

	Simulační model je před jeho prvním spuštěním nejdříve potřeba přeložit
	příkazem \texttt{make} nebo příkazem \texttt{make build} nebo příkazem
	\texttt{make ims-project}.

	Spuštění simulačního modelu se provádím příkazem \texttt{make run}. Při
	tomto spuštění je simulační model spuštěn s~výchozími hodnotami vstupních
	parametrů, které jsou popsány v~experimentu v~kapitole
	\ref{section:experiment1}. Při spouštění ostatních experimentů, viz
	kapitola \ref{sectiion:experiments}, byl simulační model spuštěn
	s~jinými hodnotami vstupních parametrů. Simulační model je možné
	spustit s~parameterm \texttt{-c} nebo \texttt{--cars}, který značí
	počet aut pro rozvoz jídel a/nebo s~parameterm \texttt{-f} nebo
	\texttt{--food}, který značí průměrný počet jídel pro rozvoz. Syntaxe
	příkazu pro spouštění simulačního modelu s~těmito parametry vypadá
	následovně (\texttt{C}~musí být kladné celé číslo a~\texttt{F}~musí
	být kladné reálné číslo): \\
	\texttt{make run [ARGS='[-c C|--cars C][ -f F| --food F]']} \\
	Např. u~experimentu v~kapitole \ref{section:experiment4} u~druhého
	řádku tabulky \ref{table:experiment4} byl experiment spuštěn příkazem: \\
	\texttt{make run ARGS='-c 42 -f 40000'}



	%%%%%%%%%%%%%%% Podstata simulačních experimentů a~jejich průběh %%%%%%%%%%%
	\section{Podstata simulačních experimentů a~jejich průběh}

	Cílem experimentů bylo nejdříve ověřit validitu modelu, případně zpětně
	upravit vstupní průměrné hodnoty, aby byl model co nejpodobnější
	skutečnosti. Další podstatou experimentů bylo demonstrovat, jak je
	systém efektivní, zjistit jeho limity a~pokusit se o~optimalizaci.
	Pomocí posledního experimentu se zjišťovalo, jestli je možné bez ohledu
	na režii systém rozšířit a~urychlit pracovní směnu.


	\subsection{Postup experimentování}

	Každý experiment spočíval ve spuštění simulace 5krát po sobě v~cyklu se
	zvolenými hodnotami. Všechny podstatné výstupní údaje se poté vložily do
	tabulky a~zprůměrovaly. V~každém experimentu také probíhala kontrola
	úspěšnosti, tedy zda se stihlo rozvézt všechno jídlo během směny. Pokud
	směna skončila v~momentě, kdy bylo alespoň 1 auto na cestě, experiment se
	označil za neúspěch. Pokud není stoprocentní úspěšnost, nelze počítat,
	že je experiment validní a~celý se považuje za neúspěšný. Nakonec se
	z~výsledků učiní závěr a~pokud nebyl experiment úspěšný, jsou na základě
	těchto informací zvoleny nové vstupní hodnoty pro další experiment. \\

	\textbf{Postup experimentování lze tedy shrnout následovně:}
	\begin{enumerate}
		\item Určení vstupních hodnot, podle zadání experimentu.
		\item Spuštění simulace.
		\item Vypsání výstupů do tabulky.
		\item Zprůměrování výsledků, zhodnocení a~učinění závěru.
	\end{enumerate}


	\subsection{Experimenty}
	\label{sectiion:experiments}

	U~jednotlivých experimentů je shrnut cíl a~smysl daného experimentu.
	Každý experiment zahrnuje výpis jeho výsledků s~patřičným popisem
	a~shrnutím závěru experimentu.


	\subsubsection{Experiment 1}
	\label{section:experiment1}

	První experiment, viz tabulka \ref{table:experiment1}, má za úkol ověřit
	validitu modelu. Používá reálné hodnoty, které byly získány při zjišťování
	informací. Průměrný počet jídel je tedy 21\,200 a~počet aut 16. Spuštěn
	byl jen jednou.

	\begin{table}[ht]
		\centering
		\begin{tabular}{|r|r|r|r|r|r|}
			\hline
			Počet jídel & Počet aut & Celkový čas [h:m]
				& Ujetá vzdálenost [km] & Spotřeba [l]
				& \textbf{Úspěch} \\ \hline

			21\,400 & 16 & 5:36 & 1\,831 & 140,9 & \textbf{100\,\%} \\ \hline
		\end{tabular}

		\caption{Experiment 1}
		\label{table:experiment1}
	\end{table}

	Z~tabulky \ref{table:experiment1} je zřejmé, že je model validní. Celkový
	čas nedosahuje nikdy 6 hodin, rozvoz má dokonce průměrně 14 minut rezervu.

	\subsubsection{Experiment 2}
	\label{section:experiment2}

	Cílem druhého experimentu, viz tabulka \ref{table:experiment2}, bylo
	zjistit limity systému ve smyslu, kolik jídel je za aktuálních podmínek
	firma schopná rozvézt za jednu směnu. Vstupní hodnota \uv{počet jídel}
	byla tedy při každém novém experimentu zvýšena o~určitou hodnotu.

	\begin{table}[ht]
		\centering
		\begin{tabular}{|r|r|r|r|r|}
			\hline
			Počet jídel & Počet aut & Celkový čas [h:m] & \textbf{Úspěch}
				& \textbf{Průměrný počet jídel} \\ \hline

			21\,400 & 16 & 5:36 & \textbf{100\,\%} & \textbf{21\,200} \\ \hline
			23\,400 & 16 & 5:39 & \textbf{100\,\%} & \textbf{23\,000} \\ \hline
			24\,500 & 16 & 5:57 & \textbf{20\,\%}  & \textbf{24\,200} \\ \hline
		\end{tabular}

		\caption{Experiment 2}
		\label{table:experiment2}
	\end{table}

	Při použití průmerného počtu jídel 23\,000 byla úspěšnost rozvozu ještě
	stoprocentní, ale při zvýšení této hodnoty o~dalších 1\,200 už z~80\,\%
	pokusů doba směny dosáhla na hodnotu 6 hodin a~některá auta stále ještě
	rozvážela. Z~experimentů vyplynulo, že při překročení hranice 24\,000 jídel
	nebylo téměř možné stihnout rovoz v~časovém intervalu.

	\subsubsection{Experiment 3}

	V~tomto experimentu, viz tabulka \ref{table:experiment3}, je záměr
	zjistit díky změně počtu aut, zda se nedá snížit spotřeba paliva a~tím
	snížit režii. Postupným odebíráním nebo přidáváním rozvážejících aut se
	bude ujetá vzálenost a~spotřeba měnit.

	\begin{table}[ht]
		\centering
		\begin{tabular}{|r|r|r|r|r|r|}
			\hline
			Počet jídel & \textbf{Počet aut} & Celkový čas [h:m]
				& Ujetá vzdálenost [km] & \textbf{Spotřeba [l]}
				& \textbf{Úspěch} \\ \hline

			21\,400 & \textbf{16} & 5:36 & 1\,831 & \textbf{140,9}
				& \textbf{100\,\%} \\ \hline

			21\,400 & \textbf{15} & 5:44 & 1\,836 & \textbf{141,3}
				& \textbf{100\,\%} \\ \hline

			21\,400 & \textbf{14} & 5:57 & 1\,801 & \textbf{138,6}
				& \textbf{40\,\%}  \\ \hline

			21\,400 & \textbf{17} & 5:30 & 1\,839 & \textbf{141,6}
				& \textbf{100\,\%} \\ \hline

			21\,400 & \textbf{25} & 3:47 & 1\,837 & \textbf{141,4}
				& \textbf{100\,\%} \\ \hline
		\end{tabular}

		\caption{Experiment 3}
		\label{table:experiment3}
	\end{table}

	Tento experiment vrací neočekávané výsledky. Při změně počtu aut, pokud je
	úspěšnost storprocentní, se vzdálenost a~tedy i~spotřeba liší jen nepatrně.
	Je zřejmé, že čím více by se udělalo pokusů, tím více by si byly hodnoty
	podobné. Lze tedy usoudit, že spotřeba není závislá na počtu aut, ale
	spíše na počtu rozvážených jídel.

	\subsubsection{Experiment 4}
	\label{section:experiment4}

	Díky poslednímu experimentu, viz tabulka \ref{table:experiment4}, je snaha
	dojít k~závěru, jestli systém ustojí i~při větších číslech. Zvýšení počtu
	průměrně rozvážených jídel na 40\,000 a~zároveň požadavek zrychení
	rozvozu o~hodinu, takže maximální přípustný čas je 5 hodin. Proměná bude
	počet aut, nezáleží na ceně.

	\begin{table}[ht]
		\centering
		\begin{tabular}{|r|r|r|r|r|r|}
			\hline
			Počet jídel & \textbf{Počet aut} & \textbf{Celkový čas [h:m]}
				& Ujetá vzdálenost [km] & Spotřeba [l]
				& \textbf{Úspěch (5h)} \\ \hline

			39\,700 & \textbf{40} & \textbf{4:34} & 3\,417 & 263,1
				& \textbf{60\,\%} \\ \hline

			39\,700 & \textbf{42} & \textbf{3:50} & 3\,397 & 262,2
				& \textbf{100\,\%} \\ \hline
		\end{tabular}

		\caption{Experiment 4}
		\label{table:experiment4}
	\end{table}

	Už po dvou experimentech se podařilo dosáhnout stoprocentní úspěšnosti. Při
	40 autech byly nějaké pokusy neúspěšné, ale při 42 autech se stihla všechna
	jídla rozvézt průměrně za 3 hodiny a~50 minut, což je dokonce o~hodinu
	rychleji než bylo požadováno. Dá se tedy tvrdit, že je systém schopen růstu.


	\subsection{Závěry experimentů}

	Bylo provedeno více než 9 experimentů v~různých podmínkách. V~průběhu
	experimentování byla zjištěna validita původního modelu. Díky dalším
	experimentům se podařilo zjistit několik užitečných informací, které
	mohou být použity ke zkvalitnění systému. Některé z~experimentů ukázaly,
	jaké jsou limitní hranice systému.

	Bylo by možné provést další experimenty, kterými by se dalo zjistit,
	jak by se systém choval, kdyby se měnila doba trvání pracovní doby.
	Další užitečné informace by se daly získat z~provedení experimentů,
	ve kterých by se změnil typ auta, které se používají pro rozvoj jídla,
	tj. změnila by se maximální kapacita auta a~jeho spotřeba. V~tomto
	případě by se ale musely nějakým vhodným způsobem spočítat a~nebo na
	základě nějakých známých faktů odhadnout ostatní parametry systému, např.
	doba nakládání auta nebo doba rozvozu jídel.

	Experimenty lze považovat s~dostatečnou věrohodností za správný zdroj
	informací, neboť se každý experiment skládal z~několika pokusů, z~nichž
	byly všechny výsledky ještě zprůměrovány.



	%%%%%%%%%%%%%%%%% Shrnutí simulačních experimentů a~závěr %%%%%%%%%%%%%%%%%%
	\section{Shrnutí simulačních experimentů a~závěr}

	Simulačními experimenty provedenými na vytvořeném modelu, viz
	kapitola \ref{section:experiment1}, byla ověřena jeho validita, protože
	při výchozích hodnotách parametrů modelu, které odpovídají získaným
	reálným faktům, se model chová analogicky k~reálnému systému.

	Studií provedenou na modelu bylo prokázáno, že systém je navržený
	tak, že za současných podmínek se vždy během pracovní směny stihne
	rozvézt všechno objednané jídlo jednotlivým zákazníkům. Z~výsledků
	simulačních experimentů dále vyplývá, že systém je za současných
	podmínek schopný rozvozu maximálně průměrně 23\,000 jídel za jeden den,
	více viz kapitola \ref{section:experiment2}. V~rámci simulačních
	experimentů bylo také zjišťěno, že při zvyšujícím se počtu aut pro
	rozvoz jídla se nesníží celková spotřeba aut, pouze se zrychlí celkový
	čas, za jaký budou jídla rozvezena.

	V~rámci tohoto projektu vznikl nástroj, který vychází z~modelu reálného
	systému firmy Freshbox. Mezi modelovaným systémem a~konceptuálním
	modelem systému je homomorfní vztah \cite[snímek 28]{IMS_slides}. Tento
	nástroj byl implementován v~jazyce C++ za použití knihovny SIMLIB.
	Při spuštění tohoto nástroje je možné měnit některé vstupní parametry
	systému a~provádět tak různé simulační experimenty. Výstupem nástroje
	jsou statistiky a~informace o~systému po skončení simulace, viz kapitola
	\ref{section:simulation_model_architecture}.



	%%%%%%%%%%%%%%%%%%%%%%%%%%%%%%%% Citace %%%%%%%%%%%%%%%%%%%%%%%%%%%%%%%%%%%%
	\clearpage
	\bibliographystyle{czechiso}
	\renewcommand{\refname}{Literatura}
	\bibliography{documentation}



	%%%%%%%%%%%%%%%%%%%%%%%%%%%%%%%% Přílohy %%%%%%%%%%%%%%%%%%%%%%%%%%%%%%%%%%%
	\clearpage
	\appendix


	\section{Petriho síť}
	\label{appendix:petri_net}

	\begin{figure}[ht]
		\centering
		\includegraphics[width=1 \linewidth]{inc/petri_net.pdf}

		\caption{Petriho síť}
	\end{figure}
\end{document}
